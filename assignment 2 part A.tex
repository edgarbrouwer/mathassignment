\documentclass[12pt]{article}
\usepackage{amsmath}

\begin{document}
\title{assignment 2 (part A) numeric simulation}
\maketitle

\section*{Question 1a}
Consider an urn with 9 balls: 3 balls are red and the remaining 6 balls are either all green or all yellow. Thus, the urn contains 3 red and 6 greens balls, or 3 red and 6 yellow balls. One ball will be drawn at random from the urn. Consider the prospects A = (R: €100; Y: €0; G: €0), B = (R: €0; Y: €100; G: €0), C = (R: €100; Y: €0; G: €100), and D = (R: €0; Y: €100; G: €100), where R, Y, and G denote the event that a red, yellow, and green ball will be drawn from the urn. \\\\
It has been found that most people prefer prospect A to B, while they simultaneously prefer prospect D to prospect C. Show that this choice pattern violates Subjective Expected Utility. \\ SEU(L) = $\sum_{i}B(ei)*u(Xi)$\\
SEU(A) = 3/9 * 100 + ½ * 6/9 * 0 + ½ * 6/9 * 0 = 33 1/3 \\
SEU(B) = 3/9 * 0 + ½ * 6/9 * 100 + ½ * 6/9 * 0 = 33 1/3 \\
SEU(C) = 3/9 * 100 + ½ * 6/9 * 0 + ½ * 6/9 * 100 = 66 2/3\\
SEU(D) = 3/9 * 0 + ½ * 6/9 * 100 + ½ * 6/9 * 100 = 66 2/3\\\\
Since SEU(A)=SEU(B) and SEU(C)=SEU(D) people should be indifferent between choosing prospect A and B and between choosing prospect C and D. 




\section*{Question 1b}
Show that the $ \alpha-MEU $ model can accommodate the majority choice.\\\\ $\alpha$-MEU( f )=$\alpha$minp$\in$C(Ep[u( f )]) + (1–$alpha$)maxp$\in$C(Ep[u( f )])\\
$\alpha= 0.8\\$
$\alpha$ - MEU(A) = 1/3 * 100 = 33 1/3\\
$\alpha$ - MEU(B) = 0.8 * (0 * 100) +  (1-0.8) * (2/3 * 100) = 13 1/3\\  
$\alpha$ -  MEU(C) = 0.8 * (1/3 * 100) + (1-0.8) * (1 * 100) = 46 2/3\\ 
$\alpha$ -  MEU(D) = 2/3 * 100 = 66 2/3\\ 
Since $ \alpha$ -MEU(A) $>$ $\alpha$ -MEU(B) and $\alpha$ -MEU(D) $>$ $\alpha$ -MEU(C) the $\alpha$ -MEU model can accommodate the majority choice. 

\section*{Question 1c}
Show that the Recursive Expected Utility model can accommodate the majority choice.\\\\ REU( f ) = E$\mu$[$\phi$(Ep$\in$C[u( f )])]\\
$\phi= x^\frac{1}{2}$\\\\
REU(A) = $(1/3 * 100)^\frac{1}{2} = 5.77$\\
REU(B) = 0.5 * (2/3 * 100 + 1/3 * 0)$^\frac{1}{2}$ + 0.5 * (0 * 100 + 2/3 * 0 + 1/3 * 0)$^\frac{1}{2}$ = 4.08\\
REU(C) = 0.5 * (1/3 * 100 + 2/3 * 100)$^\frac{1}{2}$ + 0.5 * (1/3 * 100 + 2/3 * 0)$^\frac{1}{2}$ = 7.88\\
REU(D) = 0.5 * (2/3 * 100)$^\frac{1}{2}$ + 0.5 * (2/3*100)$^\frac{1}{2}$ = 8.16\\\\
Since REU(A) $>$ REU(B) and REU(D) $>$ REU(C) recursive expected utility can accommodate the majority choice.

\section*{Question 2}

Consider the prospects E = (0.5, -3000; 0.5, 4500), F = (0.25, -6000; 0.75, 3000), G = (0.5, -1500; 0.5, 4500), and H = (0.25, -3000; 0.75, 3000). Levy and Levy (2002; Management Science) show that when asked to choose between prospects E and F, $71\%$ of subjects in the laboratory prefer E. In addition, while $76\%$ of subjects prefer to have prospect G rather prospect H. Based on these results, Levy and Levy (2002) conclude that “we reject the S-shaped [utility] function [of CPT].” Do you agree with this conclusion? Why (not)?\\\\
Step 1: comparing Expected Values:\\
EV(E) = 0.5*-3000+0.5*4500 = 750\\
EV(F) = 0.25*-6000+0.75*3000 = 750\\
EV(G) = 0.5*-1500+0.5*4500 = 1500\\
EV(H) = 0.25*-3000+0.75*3000 = 1500\\\\
Step 2: clarification:\\
We can see that the expected values of E and F are equal (and EV(G)=EV(H)). However, the vast majority of subjects in the laboratory stickily prefer E over F (or G over H). A reason for this is that people overestimate the smaller probability when it comes to gains - and thus are more risk seeking in this respect – while they are more cautious when there is a probability of losing. By this, the shape of the function is still correct when it comes to gains but not when it comes to losses. Thus, the authors are partially right with their claim.






\end{document}